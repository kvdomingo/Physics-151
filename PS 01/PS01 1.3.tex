\documentclass[12pt,a4paper,twocolumn]{article}
% The following LaTeX packages must be installed on your machine: amsmath, authblk, bm, booktabs, caption, dcolumn, fancyhdr, geometry, graphicx, hyperref, latexsym, natbib
\input{151.dat}
\usepackage{gensymb}
\usepackage{float}
\usepackage{siunitx}
\usepackage{amssymb}
\usepackage{float}
\usepackage{listings}
\PassOptionsToPackage{hyphens}{url}\usepackage{hyperref}
\usepackage[none]{hyphenat}
%\renewcommand{\familydefault}{\sfdefault}


\begin{document}

\setcounter{page}{1}

\section*{Problem 1.3}

\paragraph{(a)}
For $N = 8$ particles, there are $2^N = 256$ possible microstates, and $N + 1 = 9$ possible macrostates. Via Pascal's triangle, one can, for reasonable values of $N$ and with a bit of work, determine the number of microstates for one macrostate:

\begin{center}
\begin{tabular}{>{$}l<{$}|*{9}{c}}
\multicolumn{1}{l}{$N$} &&&&&&&&&\\\cline{1-1}
0 &1&&&&&&\\
1 &1&1&&&&&\\
2 &1&2&1&&&&\\
3 &1&3&3&1&&&\\
4 &1&4&6&4&1&&\\
5 &1&5&10&10&5&1&\\
6 &1&6&15&20&15&6&1\\
7 &1&7&21&35&35&21&7&1\\
8 &1&8&28&56&70&56&28&8&1\\\hline
\multicolumn{1}{l}{} &0&1&2&3&4&5&6&7&8\\\cline{2-10}
\multicolumn{1}{l}{} &\multicolumn{9}{c}{$n$}
\end{tabular}
\end{center}

Thus, for $n = 4$, $W(n) = 70$ and $P(n) = 70/256 \approx 27\%$. To generalize this for any $N$ and $n$ without having to explicitly write out Pascal's triangle, we can express the value for $n$ from a selection of $N$ using the combination notation $_N\mathrm{C}_n$, where

\begin{equation}
	_N\mathrm{C}_n = {N \choose n} = \frac{N!}{n!(N - n)!	}
\end{equation}

In other words, calculating for $_N\mathrm{C}_n$ yields the number of ways that $n$ particles out of $N$ can be in one partition of the box.

\paragraph{(b)}
The macrostate $n = N/2$ is much more probable than the macrostate $n = N$ because the macrostate $n = N/2$ is associated with more microstates that are indistinguishable from each other.


\end{document}