\documentclass[12pt,a4paper,twocolumn]{article}
% The following LaTeX packages must be installed on your machine: amsmath, authblk, bm, booktabs, caption, dcolumn, fancyhdr, geometry, graphicx, hyperref, latexsym, natbib
\input{151.dat}
\usepackage{gensymb}
\usepackage{float}
\usepackage{siunitx}
\usepackage{amssymb}
\usepackage{float}
\usepackage{enumerate}
\usepackage{listings}
\PassOptionsToPackage{hyphens}{url}\usepackage{hyperref}
\usepackage[none]{hyphenat}
%\renewcommand{\familydefault}{\sfdefault}


\begin{document}

\setcounter{page}{1}

\section*{Problem 1.7}
\begin{enumerate}[(a)]

\item
	\begin{itemize}

		\item \textbf{Liquid thermometer} - contains alcohol or mercury (in older iterations) that rely on thermal expansion/contraction. The meniscus of the liquid aligns with a calibrated meter on the exterior of the thermometer that corresponds to the surrounding temperature.

		\item \textbf{Digital thermometer} - contains a metal tip with a thermistor that responds to changes in temperature. The thermistor's output current depends on the surrounding temperature, and the thermometer's circuitry converts this current into a readable output.

	\end{itemize}

\item These thermometers operate on the principle of thermal equilibrium.

\item The most essential requirements for a thermometer to be useful is that it should have a contact surface with the system to be measured that is as small as possible so as to minimize the possibility of altering the system's temperature, and that the contact surface should have a low specific heat so that it is sensitive to the system's temperature changes.

\end{enumerate}

\end{document}