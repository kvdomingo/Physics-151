\documentclass[12pt,a4paper]{article}
% The following LaTeX packages must be installed on your machine: amsmath, authblk, bm, booktabs, caption, dcolumn, fancyhdr, geometry, graphicx, hyperref, latexsym, natbib
\input{151.dat}
\usepackage{gensymb}
\usepackage{float}
\usepackage{siunitx}
\usepackage{amssymb}
\usepackage{float}
\usepackage{enumerate}
\usepackage{listings}
\PassOptionsToPackage{hyphens}{url}\usepackage{hyperref}
\usepackage[none]{hyphenat}
%\renewcommand{\familydefault}{\sfdefault}


\begin{document}

\setcounter{page}{1}

\section*{Problem 1.9}
\bigskip

For a system $N = 11$ of equally spaced particles and running the simulation, the particles will move in the $+x$ direction indefinitely. By changing the velocity of one of the particles by a very tiny amount, the velocities of all the particles change greatly in response. For a sufficiently short time $t$ after the perturbation, the particles still return to their original positions, albeit only in an approximate manner, as they never line up perfectly again.

It is difficult to determine the maximum time $t$ that will allow the particles to return even in an approximate manner to their original positions due to high sensitivity to initial conditions. This indicates that the system is chaotic, and in order to fully recover the initial state by reversing time, one would have to know the initial conditions with infinite precision. Otherwise, such recovery is impossible, even in simulations. Even a brute-force search for the initial conditions is impractical since such a problem could potentially span all of $\mathbb{R}$ space.

\end{document}