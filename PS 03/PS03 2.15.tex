\documentclass[12pt,a4paper]{article}
% The following LaTeX packages must be installed on your machine: amsmath, authblk, bm, booktabs, caption, dcolumn, fancyhdr, geometry, graphicx, hyperref, latexsym, natbib
\input{151.dat}
\usepackage{gensymb}
\usepackage{amsthm}
\usepackage{float}
\usepackage{siunitx}
\usepackage{amssymb}
\usepackage{float}
\usepackage{enumerate}
\usepackage{listings}
\PassOptionsToPackage{hyphens}{url}\usepackage{hyperref}
\usepackage[none]{hyphenat}
%\renewcommand{\familydefault}{\sfdefault}


\begin{document}

\setcounter{page}{1}

\section*{Problem 2.15}
\bigskip

The given equations are:

\begin{eqnarray}
	TV^{\gamma-1} = C \label{eq:given1} \\
	PV = \nu RT \label{eq:idealgaslaw}
\end{eqnarray}

Isolating $T$ in \eqref{eq:idealgaslaw},

\begin{equation}
	T = \frac{PV}{\nu R} \label{eq:idealgastemp}
\end{equation}

Plugging this into \eqref{eq:given1},

\begin{align}
	\frac{PV}{\nu R} V^{\gamma - 1} = C \nonumber \\
	PV^{\gamma} = C\nu R = C
\end{align}

Therefore,

\begin{equation}\label{eq:answer1}
\boxed{	PV^\gamma = C	}
\end{equation}

Similarly, isolating $V$ from \eqref{eq:idealgaslaw},

\begin{equation}\label{eq:idealgasvol}
	V = \frac{\nu RT}{P}
\end{equation}

Plugging this into \eqref{eq:answer1},

\begin{align}
	P\left( \frac{\nu RT}{P} \right)^\gamma & = C \nonumber \\
	\left( \nu RT \right)^\gamma P^{1-\gamma} & = C \nonumber \\
	T^\gamma P^{1-\gamma} & = \frac{C}{\left( \nu R \right)^\gamma} \nonumber \\
	T^\gamma P^{1-\gamma} & = C
\end{align}

Raising both sides to $1/\gamma$,

\begin{eqnarray}
	\left( T^\gamma P^{1-\gamma} \right)^{\frac{1}{\gamma}} = C^{\frac{1}{\gamma}} \nonumber \\
\boxed{	TP^{\frac{1 - \gamma}{\gamma}} = C} \label{eq:answer2}
\end{eqnarray}

\end{document}