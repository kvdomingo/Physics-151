\documentclass[12pt,a4paper]{article}
% The following LaTeX packages must be installed on your machine: amsmath, authblk, bm, booktabs, caption, dcolumn, fancyhdr, geometry, graphicx, hyperref, latexsym, natbib
\input{151.dat}
\usepackage{gensymb}
\usepackage{amsthm}
\usepackage{float}
\usepackage{siunitx}
\usepackage{amssymb}
\usepackage{float}
\usepackage{enumerate}
\usepackage{listings}
\usepackage{mathtools}
\PassOptionsToPackage{hyphens}{url}\usepackage{hyperref}
\usepackage[none]{hyphenat}
\usepackage{physics}
%\renewcommand{\familydefault}{\sfdefault}


\begin{document}

\setcounter{page}{1}

\section*{Problem 2.20}
\bigskip

From the given equation,

\begin{equation}\label{eq:cop}
	\textrm{COP} = \frac{Q_{hot}}{W}
\end{equation}

Since we want to heat a room surrounded by a cooler environment, the work done is positive:

\begin{align}
	\textrm{COP} &= \frac{Q_{hot}}{Q_{hot} - Q_{cold}} \\
	&= \frac{N k_B T_{hot} \ln{\left( \frac{V_f}{V_i} \right)}}{Nk_B T_{hot}\ln{\left( \frac{V_f}{V_i} \right)} - Nk_B T_{cold} \ln{\left( \frac{V_f}{V_i} \right)}} \nonumber  \\
	\Aboxed{\textrm{COP} &= \frac{T_{hot}}{T_{hot} - T_{cold}}} \label{eq:answer1}
\end{align}

Plugging the given conditions $T_{hot} = 296$K and $T_{cold} = 273$K into \eqref{eq:answer1},

\begin{align}
	\textrm{COP} &= \frac{296}{296-273} \nonumber \\
	\Aboxed{\textrm{COP} &= 12.9} \label{eq:answer2}
\end{align}

The denominator is larger for large temperature differences. The heater is more efficient in regions with mild winters.

\end{document}