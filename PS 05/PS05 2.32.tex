\documentclass[12pt,a4paper]{article}
% The following LaTeX packages must be installed on your machine: amsmath, authblk, bm, booktabs, caption, dcolumn, fancyhdr, geometry, graphicx, hyperref, latexsym, natbib
\input{151.dat}
\usepackage{gensymb}
\usepackage{amsthm}
\usepackage{float}
\usepackage{siunitx}
\usepackage{amssymb}
\usepackage{float}
\usepackage{enumerate}
\usepackage{listings}
\usepackage{mathtools}
\PassOptionsToPackage{hyphens}{url}\usepackage{hyperref}
\usepackage[none]{hyphenat}
\usepackage{physics}
%\renewcommand{\familydefault}{\sfdefault}


\begin{document}

\setcounter{page}{1}

\section*{PS 22: Problem 2.32}
\bigskip

\begin{enumerate}[(a)]

\item The Legendre transform of a function $f(x)$ is given by

\begin{equation}\label{eq:legendre-def}
	\mathcal{G}\qty[m(x)] = f(x) - xm
\end{equation}

where $m = f'(x)$.

Let $f = x^3$. It's Legendre transform is

\begin{align}
	\mathcal{G}\qty[m(x)] &= x^3 - x(3x^2) \nonumber \\
	&= -2x^3
\end{align}

But this must be expressed only in terms of $m$. Using $m = f'(x)$,

\begin{align}
	m &= 3x^2 \nonumber \\
	x &= \sqrt{\frac{m}{3}}
\end{align}

Thus,

\begin{equation}\label{eq:answer-a}
	\boxed{
		\mathcal{G}\qty[x^3] = -2\qty(\frac{m}{3})^{\frac{3}{2}}
	}
\end{equation}

\item Let $f(x) = x$. Its Legendre transform is

\begin{align}
	\mathcal{G}\qty[m(x)] &= f(x) - xm \nonumber \\
	&= x - x(1) \nonumber \\
	\Aboxed{	
		\mathcal{G}\qty[x] &= 0 \label{eq:answer-b1}
	}
\end{align}

Let $f(x) = \sin(x)$. Its Legendre transform is

\begin{align}
	\mathcal{G}[m(x)] &= \sin(x) - x\cos(x) \nonumber \\
	x &= \cos^{-1}(m) \nonumber \\
	\mathcal{G}[m(x)] &= \sin(\cos^{-1}(m)) - m\cos^{-1}(m) \label{eq:sincos}
\end{align}

We can simplify this using the property

\begin{equation}
	\sin(\cos^{-1}(m)) = \sqrt{1 - m^2} \label{eq:cosproperty}
\end{equation}

Using this, we rewrite \eqref{eq:sincos} as

\begin{equation}\label{eq:answer-b2}
	\boxed{	
		\mathcal{G}[\sin(x)] = \sqrt{1-m^2} - m\cos^{-1}(m)
	}
\end{equation}

\end{enumerate}

\end{document}