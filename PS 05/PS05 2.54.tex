\documentclass[12pt,a4paper,twocolumn]{article}
% The following LaTeX packages must be installed on your machine: amsmath, authblk, bm, booktabs, caption, dcolumn, fancyhdr, geometry, graphicx, hyperref, latexsym, natbib
\input{151.dat}
\usepackage{gensymb}
\usepackage{amsthm}
\usepackage{float}
\usepackage{siunitx}
\usepackage{amssymb}
\usepackage{float}
\usepackage{enumerate}
\usepackage{listings}
\usepackage{mathtools}
\PassOptionsToPackage{hyphens}{url}\usepackage{hyperref}
\usepackage[none]{hyphenat}
\usepackage{physics}
%\renewcommand{\familydefault}{\sfdefault}


\begin{document}

\setcounter{page}{1}

\section*{PS 24: Problem 2.54}
\bigskip

\begin{enumerate}[(a)]

\item The radius of a black hole depends on the following:

\begin{align}
	F_G &= G\frac{M}{R^2} = \mathrm{\qty[\frac{m^3}{kg\cdot s^2}]\qty[\frac{kg}{m^2}]} \label{eq:gravity} \\
	c &= \mathrm{constant} = \mathrm{\qty[\frac{m}{s}]} \label{eq:light} \\
	M &= \mathrm{constant} = \mathrm{\qty[kg]} \label{eq:mass}
\end{align}

We can estimate the radius $R$ of the black hole by dimensional analysis of \eqref{eq:gravity}-\eqref{eq:mass}. We have

\begin{equation}
	R \approx \frac{GM}{c^2} = \mathrm{\frac{\qty[\frac{m^3}{kg\cdot s^2}]\qty[kg]}{\qty[\frac{m}{s}]^2}} = \mathrm{\qty[m]}
\end{equation}

Therefore, the radius of a black hole is

\begin{equation}\label{eq:answer-a}
	\boxed{
		R \approx \frac{GM}{c^2}
	}
\end{equation}

\item Taking photons with wavelength $\lambda$

\begin{equation}\label{eq:wavelen}
	\lambda = 2R = 2\frac{GM}{c^2}
\end{equation}

whose energy $E_\gamma$ is

\begin{align}
	E_\gamma &= \frac{hc}{\lambda} \\
	&= \frac{hc^3}{2GM} \label{eq:photonenergy}
\end{align}

and the black hole's total energy $E$

\begin{equation}\label{eq:totalenergy}
	E = Mc^2%
\end{equation}

The momentum of a photon is given by

\begin{equation}\label{eq:photonmomentum}
	p_\gamma = \frac{h}{\lambda}
\end{equation}

If we consider the system to behave classically and non-relativistically, then we can recall the classical momentum

\begin{equation}\label{eq:classicalmomentum}
	p = mv
\end{equation}

and from this, we see that we can divide \eqref{eq:photonmomentum} by the photon's velocity $c$ to get its mass

\begin{equation}\label{eq:photonmass}
	m_\gamma = \frac{h}{c\lambda}
\end{equation}

If we assume that the entropy of the black hole is of order $Nk$, where $N$ is the number of particles in the black hole, and that all the particles are photons, we can estimate this entropy to be

\begin{equation}\label{eq:bhentropy}
	S \approx Nk_B
\end{equation}

where $k_B$ is Boltzmann's constant. If the total energy of the black hole is given by \eqref{eq:totalenergy}, the number of photons is

\begin{equation}
	N = \frac{E}{E_\gamma} = \frac{2GM^2}{hc}
\end{equation}

Plugging this into \eqref{eq:bhentropy},

\begin{equation}\label{eq:answer-b}
	\boxed{
		S = k_B \frac{2GM^2}{hc}
	}
\end{equation}

The entropy for a black hole of one solar mass is $\boxed{S \approx 4 \times 10^{52}$  $\mathrm{J\cdot K}}$.

\item Recall the surface area of a sphere:

\begin{equation}\label{eq:spherearea}
	A = 4\pi R^2
\end{equation}

Plugging in \eqref{eq:answer-a} into this,

\begin{equation}
	A = \frac{4\pi M^2}{c^4}
\end{equation}

Plugging this into \eqref{eq:answer-b},

\begin{equation}
	\boxed{
		S = k_B \frac{\pi Gc^3}{2h}A
	}
\end{equation}

Thus, entropy increases when black holes coalesce.

\item Using \eqref{eq:totalenergy} to express \eqref{eq:answer-b} in terms of $E$,

\begin{equation}\label{eq:entropyinE}
	S = k_B \frac{2GE^2}{hc^5}
\end{equation}

From the equation of state of $S$,

\begin{equation}\label{eq:ftr}
	\dd{E} = T\dd{S} - P\dd{V} + \mu\dd{N}
\end{equation}

we see that the natural variable $T$ can be expressed as

\begin{equation}\label{eq:naturaltemp}
	T = \qty(\pdv{S}{E})^{-1}
\end{equation}

Performing the differentiation on \eqref{eq:entropyinE},

\begin{equation}\label{eq:diffS}
	T = \qty(k_B \frac{4GE}{hc^5})^{-1}
\end{equation}

We express \eqref{eq:diffS} once again in terms of $M$ so that

\begin{equation}\label{eq:answer-d1}
	\boxed{
		T = \frac{hc^3}{4k_B GM}
	}
\end{equation}

So the temperature for a black hole of one solar mass is $\boxed{T \approx 2\times 10^{-6} \textrm{ K}}$.

Expressing \eqref{eq:answer-d1} in terms of $R$ in \eqref{eq:answer-a},

\begin{equation}\label{eq:answer-d2}
	\boxed{
		T = \frac{hc}{4k_B R}
	}
\end{equation}

\end{enumerate}

\end{document}