\documentclass[12pt,a4paper,twocolumn]{article}
% The following LaTeX packages must be installed on your machine: amsmath, authblk, bm, booktabs, caption, dcolumn, fancyhdr, geometry, graphicx, hyperref, latexsym, natbib
\input{151.dat}
\usepackage{gensymb}
\usepackage{amsthm}
\usepackage{float}
\usepackage{siunitx}
\usepackage{amssymb}
\usepackage{float}
\usepackage{enumerate}
\usepackage{listings}
\usepackage{mathtools}
\PassOptionsToPackage{hyphens}{url}\usepackage{hyperref}
\usepackage[none]{hyphenat}
\usepackage{physics}
%\renewcommand{\familydefault}{\sfdefault}


\begin{document}

\setcounter{page}{1}

\section*{PS 25: Problem 2.57}
\bigskip

\begin{enumerate}[(a)]

\item The Van der Waals equation of state is given by

\begin{equation}\label{eq:vanderwaals}
	\qty(P + \frac{N^2}{V^2})\qty(V-Nb) = NkT
\end{equation}

Since the density $\rho = \frac{N}{V} \ll 1$, \eqref{eq:vanderwaals} can be approximated as

\begin{equation}\label{eq:vdwapprox}
	P\qty(V-Nb) = NkT
\end{equation}

Rearranging terms to isolate $P$,

\begin{equation}\label{eq:pressure}
	P = \frac{NkT}{V-Nb}
\end{equation}

The work done on the gas is

\begin{equation}\label{eq:work}
	W = -\int_{V_1}^{V_2} P \dd{V}
\end{equation}

Plugging in \eqref{eq:pressure} into \eqref{eq:work},

\begin{align}
	W_{VDW} &= -\int_{V_1}^{V_2} \frac{NkT}{V-Nb} \dd{V} \nonumber \\
	&= -NkT \int_{V_1}^{V_2} \frac{\dd{V}}{V-Nb}
\end{align}

Let $u \equiv V - Nb$, $\dd{u} \equiv \dd{V}$.

\begin{align}
	W_{VDW} &= -NkT \int_{V_1-Nb}^{V_2-Nb} \frac{\dd{u}}{u} \nonumber \\
	&= -NkT \eval{\ln(u)}_{V_1-Nb}^{V_2-Nb} \nonumber \\
	&= -NkT \qty[\ln(V_2-Nb) - \ln(V_1-Nb)] \nonumber \\
	\Aboxed{
		W_{VDW} &= NkT \ln\qty[\frac{V_1-Nb}{V_2-Nb}] \label{eq:answer-a}
	}
\end{align}


\item Assuming the gas is ideal and subjected to the same conditions, recall the ideal gas law and isolate $P$ once again:

\begin{align}
	PV &= NkT \label{eq:idealgas} \\
	P &= \frac{NkT}{V} \label{eq:idealpressure}
\end{align}

The work done on the gas is

\begin{align}
	W_{ideal} &= -\int_{V_1}^{V_2} \frac{NkT}{V} \dd{V} \nonumber \\
	&= -NkT \eval{\ln(V)}_{V_1}^{V_2} \nonumber \\
	&= -NkT \qty[\ln(V_2) - \ln(V_1)] \nonumber \\
	\Aboxed{
		W_{ideal} &= NkT\ln\qty(\frac{V_1}{V_2}) \label{eq:answer-b}
	}
\end{align}

Getting the difference of \eqref{eq:answer-a} and \eqref{eq:answer-b},

\begin{align}
	W_{VDW} - W_{ideal} &= NkT\ln\qty[\frac{V_1-Nb}{V_2-Nb}] - NkT\ln\qty[\frac{V_1}{V_2}] \nonumber \\
	\Aboxed{	
		W_{VDW} - W_{ideal} &= NkT\ln\qty[\frac{\qty(V_1-Nb)V_2}{\qty(V_2-Nb)V_1}]
	} 
\end{align}

This implies that when $V_2 > V_1$ the $\ln$ term is less than zero and contributes a negative term: the work done by a non-ideal gas is less than the work done by an ideal gas. This may be due to the factor of $Nb$ subtracted from the terms in the Van der Waals equation which accounts for the size of the particles.

\end{enumerate}

\end{document}