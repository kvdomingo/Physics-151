\documentclass[12pt,a4paper]{article}
% The following LaTeX packages must be installed on your machine: amsmath, authblk, bm, booktabs, caption, dcolumn, fancyhdr, geometry, graphicx, hyperref, latexsym, natbib
\input{151.dat}
\usepackage{gensymb}
\usepackage{amsthm}
\usepackage{float}
\usepackage{siunitx}
\usepackage{amssymb}
\usepackage{float}
\usepackage{enumerate}
\usepackage{listings}
\usepackage{mathtools}
\PassOptionsToPackage{hyphens}{url}\usepackage{hyperref}
\usepackage[none]{hyphenat}
\usepackage{physics}
%\renewcommand{\familydefault}{\sfdefault}


\begin{document}

\setcounter{page}{1}

\section*{PS 28: Problem 3.34}
\bigskip

\begin{enumerate}[(a)]

\item Assuming that the gas is dilute and that position of a molecule is independent of the position of the others', the probability of finding a particular molecule in $V_1$ is

\begin{equation}\label{eq:answer-a}
	\boxed{
	p = \frac{1}{2}
	}
\end{equation}

\item Given that the total number of particles is $N = N_1 + N_2$, the probability of finding $N_1$ particles in $V_1$ can be calculated using the binomial distribution:

\begin{align}
	P_N(n) &= \frac{N!}{n!\qty(N-n)!} 2^{-N} \label{eq:binomial} \\
	\Aboxed{
		P_N(N_1) &= \frac{N!}{N_1!\qty(N-N_1)!} 2^{-N}
	} \label{eq:answer-b1} \\
	P_N(N_2) &= \frac{N!}{N_2!\qty(N-N_2)!} 2^{-N} \nonumber \\
	&= \frac{N!}{\qty(N-N_1)!\qty[N-\qty(N-N_1)]!}2^{-N} \nonumber \\
	&= \frac{N!}{\qty(N-N_1)! N_1!}2^{-N} \nonumber \\
	\Aboxed{
		P_N(N_2) &= P_N(N_1)
	} \label{eq:answer-b2}
\end{align}

\item The average number of molecules in each part is

\begin{align}
	\bar{n} &= pN \label{eq:mean} \\
	\Aboxed{
		\bar{n} &= \frac{1}{2}N
	} \label{eq:answer-c}
\end{align}

\item The relative fluctuation is

\begin{align}
	\overline{\qty(\Delta n)^2} &= p\qty(1-p)N \label{eq:fluc} \\
	&= \frac{1}{2}\qty(1 - \frac{1}{2})N \nonumber \\
	\Aboxed{
		\overline{\qty(\Delta n)^2} &= \frac{1}{4}N
	} \label{eq:answer-d}
\end{align}

\end{enumerate}

\end{document}