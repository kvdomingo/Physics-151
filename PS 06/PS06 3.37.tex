\documentclass[12pt,a4paper,twocolumn]{article}
% The following LaTeX packages must be installed on your machine: amsmath, authblk, bm, booktabs, caption, dcolumn, fancyhdr, geometry, graphicx, hyperref, latexsym, natbib
\input{151.dat}
\usepackage{gensymb}
\usepackage{amsthm}
\usepackage{float}
\usepackage{siunitx}
\usepackage{amssymb}
\usepackage{float}
\usepackage{enumerate}
\usepackage{listings}
\usepackage{mathtools}
\PassOptionsToPackage{hyphens}{url}\usepackage{hyperref}
\usepackage[none]{hyphenat}
\usepackage{physics}
%\renewcommand{\familydefault}{\sfdefault}


\begin{document}

\setcounter{page}{1}

\section*{PS 30: Problem 3.37}
\bigskip

With the given

\begin{gather}
	\ln A = \ln P_n(n = \bar{n}) \label{givenln} \\
	P_N(n) = Ae^{-\frac{1}{2\sigma^2}(n - \bar{n})^2} \label{eq:givenfunc} \\
	P_N(n) = \frac{N!}{n!(N-n)!}p^N(1-p)^{N-n} \label{eq:binom} \\
	\ln N! = N\ln N - N + \frac{1}{2}\ln\qty(2\pi N) \label{eq:stirling}
\end{gather}

Take the $\ln$ of both sides of \eqref{eq:binom}

\begin{align}
	\ln P_N(n) &= \ln\qty[\frac{N!}{n!(N-n)!}p^N(1-p)^{N-n}] \nonumber \\
	\ln P_N(n) &= \ln N! - \ln n! - \ln\qty(N-n)! \nonumber \\
	& \quad + N\ln p + (N-n)\ln q
\end{align}

Since $n$ is constant, we can write it interchangeably with $\bar{n}$:

\begin{align}
	\ln P_N(n) &= \ln N! - \ln \bar{n}! - \ln\qty(N-\bar{n})! \nonumber \\
	& \quad + N\ln p + (N-\bar{n})\ln q
\end{align}

Recall that $\bar{n} = Np$. We can write

\begin{align}
	\ln P_N(n) &= \ln N! - \ln\qty(Np)! - \ln\qty(N - Np)! \nonumber \\
	& \quad + N \ln p + (N - Np)\ln q \nonumber \\
	\ln P_N(n) &= \ln N! - \ln\qty(Np)! - \ln\qty(N - Np)! \nonumber \\
	& \quad + N \ln p + [N(1-p)]\ln q
\end{align}

Using Stirling's approximation in \eqref{eq:stirling},

\begin{align}
	\ln P_N(n) &= N \ln N - N + \frac{1}{2}\ln\qty(2\pi N) \nonumber \\
	&- Np\ln\qty(Np) + Np - \frac{1}{2}\qty(2\pi Np) \nonumber \\
	&- N(1-p)\ln\qty[N(1-p)] + N(1-p) \nonumber \\
	&- \frac{1}{2}\ln\qty[2\pi N(1-p)] \nonumber \\
	&+ Np\ln p + N(1-p)\ln q \\
	\ln P_N(n) &= N \ln N - N + \frac{1}{2}\ln\qty(2\pi N) \nonumber \\
	&- Np\ln\qty(Np) + Np - \frac{1}{2}\qty(2\pi Np) \nonumber \\
	&- Nq\ln\qty(Nq) + Nq \nonumber \\
	&- \frac{1}{2}\ln\qty[2\pi Nq] \nonumber \\
	&+ Np\ln p + Nq\ln q \\
	\ln P_N(n) &= N\ln N - N\frac{1}{2}\ln\qty(2\pi N) \nonumber \\
	&- Np\ln N - Np \ln p - \frac{1}{2}\ln p \nonumber \\
	&- \frac{1}{2}\ln\qty(2\pi N) - Nq\ln N - Nq\ln q \nonumber \\
	&+ Nq -\frac{1}{2}\ln\qty(2\pi Nq) + Np\ln p + Nq\ln q \\
	\ln P_N(n) &= -\frac{1}{2}\ln\qty(2\pi Npq)
\end{align}

But $\ln P_N(n) = \ln A$

\begin{align}
	\ln A &= -\frac{1}{2}\ln\qty(2\pi Npq) \\
	A &= \frac{1}{\sqrt{2\pi Npq}}
\end{align}

But $Npq = \sigma^2$, so

\begin{equation}
	\boxed{
		A = \frac{1}{\sqrt{2\pi\sigma^2}}
	}
\end{equation}

\end{document}