\documentclass[12pt,a4paper,twocolumn]{article}
% The following LaTeX packages must be installed on your machine: amsmath, authblk, bm, booktabs, caption, dcolumn, fancyhdr, geometry, graphicx, hyperref, latexsym, natbib
\input{151.dat}
\usepackage{gensymb}
\usepackage{amsthm}
\usepackage{float}
\usepackage{siunitx}
\usepackage{amssymb}
\usepackage{float}
\usepackage{enumerate}
\usepackage{listings}
\usepackage{mathtools}
\PassOptionsToPackage{hyphens}{url}\usepackage{hyperref}
\usepackage[none]{hyphenat}
\usepackage{physics}
%\renewcommand{\familydefault}{\sfdefault}


\begin{document}

\setcounter{page}{1}

\section*{PS 31: Problem 3.39}
\bigskip

The random variable $x$ has the probability density

\begin{equation}\label{eq:given}
	p(x) =
	\begin{cases}
		Ae^{-\lambda x} & 0 \leq x \leq \infty \\
		0 & x < 0
	\end{cases}
\end{equation}


\begin{enumerate}[(a)]

\item To get the normalization constant $A$ of \eqref{eq:given}, we take its integral over all space. Since the function is zero if $x < 0$, we consider only the area where it is non-zero and equate it to unity:

\begin{align}
	1 &= \int_0^\infty Ae^{-\lambda x} \dd{x} \\
	1 &= A\int_0^\infty e^{-\lambda x}\dd{x} \nonumber \\
	1 &= A\eval{\frac{e^{-\lambda x}}{-\lambda}}_0^\infty \nonumber \\
	1 &= \frac{A}{-\lambda}\qty[e^{-\lambda\infty} - e^0] \nonumber \\
	1 &= \frac{A}{-\lambda}\qty[-1] \nonumber \\
	\Aboxed{
		A &= \lambda
	} \label{eq:answer-a}
\end{align}

\eqref{eq:given} now becomes

\begin{equation}\label{eq:given-normed}
	p(x) =
	\begin{cases}
		\lambda e^{-\lambda x} & 0 \leq x \leq \infty \\
		0 & x < 0
	\end{cases}
\end{equation}

\item The mean value of $x$ is

\begin{align}
	\ev{x} &= \int_0^\infty x p(x) \dd{x} \\
	&= \int_0^\infty x\lambda e^{-\lambda x} \dd{x} \nonumber \\
	&= \lambda \int_0^\infty xe^{-\lambda x} \dd{x}
\end{align}

Using integration by parts, let $u \equiv x$ and $\dd{v} \equiv e^{-\lambda x}\dd{x}$. Consequently, $\dd{u} \equiv \dd{x}$ and $v \equiv -e^{-\lambda x}/a$. We have

\begin{equation}\label{eq:ibp}
	\ev{x} = \lambda\qty[\eval{-\frac{xe^{-\lambda x}}{\lambda}}_0^\infty + \frac{1}{\lambda}\int_0^\infty e^{-\lambda x}\dd{x}]
\end{equation}

The integral in \eqref{eq:ibp} follows the same solution as in part (a), and we end up with

\begin{align}
	\ev{x} &= \lambda \eval[-\frac{xe^{-\lambda x}}{\lambda} - \frac{e^{-\lambda x}}{\lambda^2}|_0^\infty \nonumber \\
	&= -\eval[\frac{\lambda xe^{-\lambda x} + e^{-\lambda x}}{\lambda}|_0^\infty
\end{align}

Substituting $\infty$ directly into the first term causes it to blow up. However, upon examining the terms involved we see that $\dv[2]{x}x < \dv[2]{x}e^{-\lambda x}$. Thus, we can say that the exponential term approaches zero faster than $x$ approaches infinity, so the exponential term dominates. Therefore,

\begin{align}
	\ev{x} &= -\qty[\frac{e^{-\lambda\infty} - e^0}{\lambda}] \nonumber \\
	&= -\frac{0 - 1}{\lambda} \nonumber \\
	\Aboxed{
		\ev{x} &= \frac{1}{\lambda}
	} \label{eq:answer-b}
\end{align}

The most probable value of $\boxed{x = 0}$.

\item The mean value of $x^2$ is

\begin{equation}
	\ev{x^2} = \lambda\int_0^\infty x^2 e^{-\lambda x}\dd{x}
\end{equation}

Once again, integrate by parts, letting $u \equiv x^2$ and $\dd{v} \equiv e^{\lambda x}\dd{x}$. So $\dd{u} \equiv 2x\dd{x}$ and $v \equiv -e^{-\lambda x}/a$. We have

\begin{equation}
	\ev{x^2} = \lambda\qty[\eval{-\frac{x^2 e^{-\lambda x}}{\lambda}}_0^\infty + \frac{2}{\lambda}\int_0^\infty xe^{-\lambda x}\dd{x}]
\end{equation}

Following the steps in part (b), the integral can be simplified as

\begin{align}
	\ev{x^2} &= \lambda \qty[\eval{-\frac{2xe^{-\lambda x}}{\lambda^2} - \frac{x^2 e^{-\lambda x}}{\lambda}}_0^\infty + \frac{2}{\lambda}\int_0^\infty e^{-\lambda x}\dd{x}] \nonumber \\
	&= \lambda \eval[-\frac{2e^{-\lambda x}}{\lambda^3} - \frac{2xe^{-\lambda x}}{\lambda^2} - \frac{x^2 e^{-\lambda x}}{\lambda}|_0^\infty \nonumber \\
	&= \lambda\qty[\frac{2}{\lambda^3} + 0 + 0] \nonumber \\
	\Aboxed{
		\ev{x^2} &= \frac{2}{\lambda^2}
	} \label{eq:answer-c}
\end{align}

\item For $\lambda = 1$, the probability that a measurement of $x$ yields a value $\in \qty(1,2)$ is

\begin{align}
	\Pr(1 < x < 2) &= \int_1^2 p(x) \dd{x} \\
	&= \int_1^2 e^{-x} \dd{x} \nonumber \\
	&= \eval{-e^{-x}}_1^2 \nonumber \\
	&= -e^{-2} + e^{-1} \nonumber \\
	\Aboxed{
		\Pr(1<x<2) &= \frac{1}{e} - \frac{1}{e^2} \approx 0.23
	} \label{eq:answer-d}
\end{align}

\item For $\lambda = 1$, the probability that a measurement of $x$ yields a value $< 0.3$ is

\begin{align}
	\Pr(x<0.3) &= \int_0^{0.3} p(x) \dd{x} \\
	&= \int_0^{0.3} e^{-x} \dd{x} \nonumber \\
	&= \eval{-e^{-x}}_0^{0.3} \nonumber \\
	&= -e^{-0.3} + e^{-0} \nonumber \\
	\Aboxed{
		\Pr(x<0.3) &= 1 - \frac{1}{e^{0.3}} \approx 0.26
	} \label{eq:answer-e}
\end{align}

\end{enumerate}

\end{document}