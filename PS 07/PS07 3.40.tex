\documentclass[12pt,a4paper,twocolumn]{article}
% The following LaTeX packages must be installed on your machine: amsmath, authblk, bm, booktabs, caption, dcolumn, fancyhdr, geometry, graphicx, hyperref, latexsym, natbib
\input{151.dat}
\usepackage{gensymb}
\usepackage{amsthm}
\usepackage{float}
\usepackage{siunitx}
\usepackage{amssymb}
\usepackage{float}
\usepackage{enumerate}
\usepackage{listings}
\usepackage{mathtools}
\PassOptionsToPackage{hyphens}{url}\usepackage{hyperref}
\usepackage[none]{hyphenat}
\usepackage{physics}
%\renewcommand{\familydefault}{\sfdefault}


\begin{document}

\setcounter{page}{1}

\section*{PS 32: Problem 3.40}
\bigskip

Consider the probability density function

\begin{equation}\label{eq:given}
	p(v_x) = \qty(\frac{a}{\pi})^{\frac{3}{2}} e^{-av_x^2}
\end{equation}

for the velocity of a particle in the $x$-direction. The probability densities for $v_y$ and $v_z$ have the same form. Each of the three velocity components can range from $-\infty$ to $+\infty$, and $a$ is a constant.

\begin{enumerate}[(a)]

\item The velocity vector probability density is given by

\begin{align}
	p(\vec{v}) &= p(v_x + v_y + v_z) \\
	&= \iiint_{-\infty}^{+\infty} \qty(\frac{a}{\pi})^{\frac{3}{2}} e^{-a(v_x^2 + v_y^2 + v_z^2)} \dd{\vec{v}} \label{eq:combined-integral} \\
	\begin{split}
		1 &= \qty(\frac{a}{\pi})^{\frac{3}{2}}\qty(\int_{-\infty}^{+\infty} e^{-av_x^2} \dd{v_x}) \cdot \\
		& \quad \qty(\int_{-\infty}^{+\infty} e^{-av_y^2} \dd{v_y}) \cdot \\
		& \quad \qty(\int_{-\infty}^{+\infty} e^{-av_z^2} \dd{v_z}) \label{eq:normcondition}
	\end{split}
\end{align}

where \eqref{eq:normcondition} follows from the normalization condition. From (GT 3.123),

\begin{equation}\label{eq:given-integral}
	\int_0^\infty e^{-au^2}\dd{u} = \frac{1}{2}\sqrt{\frac{\pi}{a}}
\end{equation}

The multiple integrals defined in \eqref{eq:normcondition} can be expressed in a form similar to \eqref{eq:given-integral} by noting that they are symmetric about zero, so that they can be recast in the form

\begin{equation}\label{eq:symmetric-integral}
	\int_{-\infty}^{+\infty} e^{-av_x^2}\dd{v_x} = 2\int_0^\infty e^{-av_x^2} \dd{v_x}
\end{equation}

\eqref{eq:normcondition} now becomes

\begin{align}
	1 &= \qty(\frac{a}{\pi})^{\frac{3}{2}} \qty[\qty(\sqrt{\frac{\pi}{a}}) \qty(\sqrt{\frac{\pi}{a}}) \qty(\sqrt{\frac{\pi}{a}})] \nonumber \\
	1 &= \qty(\frac{a}{\pi})^{\frac{3}{2}} \qty(\sqrt{\frac{\pi}{a}})^3 \nonumber \\
	1 &= \qty(\frac{a}{\pi})^{\frac{3}{2}} \qty(\frac{\pi}{a})^{\frac{3}{2}} \nonumber \\
	\Aboxed{
		1 &= 1
	} \label{eq:answer-a}
\end{align}

Therefore, $p(\vec{v})$ is already normalized.

\item To get the probability that a particle has a velocity $\in (v_x, v_x + \dd{v_x}) \cup (v_y, v_y + \dd{y}) \cup (v_z, v_z + \dd{z})$, we evaluate the indefinite form of \eqref{eq:combined-integral}.

\begin{align}
	\begin{split}
		\Pr(\vec{v} \leq \vec{v}' \leq \vec{v}+\dd{\vec{v}}) = \\
		\iiint \qty(\frac{a}{\pi})^{\frac{3}{2}} e^{-a\cdot\vec{v}^2} \dd{\vec{v}}
	\end{split} \label{eq:indefinite-combined} \\
	\begin{split}
		&= \qty(\frac{a}{\pi})^{\frac{3}{2}}\qty(\int e^{-av_x^2}\dd{v_x}) \cdot \\
		&\qquad \qty(\int e^{-av_y^2}\dd{v_y}) \cdot \\
		&\qquad \qty(\int e^{-av_z^2}\dd{v_z})
	\end{split} \label{eq:indefinite-split}
\end{align}

\item To get the probability of all the components of $\vec{v}\geq 0$ simultaneously, we evaluate \eqref{eq:combined-integral} over the non-negative range:

\begin{align}
	\Pr(\vec{v} \geq 0) &= \iiint_0^\infty \qty(\frac{a}{\pi})^{\frac{3}{2}} e^{-a\cdot\vec{v}}\dd{\vec{v}} \\
	\begin{split}
		&= \qty(\frac{a}{\pi})^{\frac{3}{2}}\qty(\int_0^{+\infty} e^{-av_x^2} \dd{v_x}) \cdot \\
		& \quad \qty(\int_0^{+\infty} e^{-av_y^2} \dd{v_y}) \cdot \\
		& \quad \qty(\int_0^{+\infty} e^{-av_z^2} \dd{v_z})
	\end{split} \nonumber \\
	&= \qty(\frac{a}{\pi})^{\frac{3}{2}}\qty[\frac{1}{2}\qty(\sqrt{\frac{\pi}{a}})]^3 \nonumber \\
	\Aboxed{
		\Pr(\vec{v} \geq 0) &= \frac{1}{8}
	} \label{eq:answer-c}
\end{align}

\end{enumerate}

\end{document}