\documentclass[12pt,a4paper]{article}
% The following LaTeX packages must be installed on your machine: amsmath, authblk, bm, booktabs, caption, dcolumn, fancyhdr, geometry, graphicx, hyperref, latexsym, natbib
\input{151.dat}
\usepackage{gensymb}
\usepackage{amsthm}
\usepackage{float}
\usepackage{siunitx}
\usepackage{amssymb}
\usepackage{float}
\usepackage{enumerate}
\usepackage{listings}
\usepackage{mathtools}
\PassOptionsToPackage{hyphens}{url}\usepackage{hyperref}
\usepackage[none]{hyphenat}
\usepackage{physics}
%\renewcommand{\familydefault}{\sfdefault}


\begin{document}

\setcounter{page}{1}

\section*{PS 38: Problem 4.14}
\bigskip

From GT 4.42, the number of microstates accessible to a gas molecule in a 1-liter box is

\begin{equation}
	\Gamma(E) = \frac{4\pi}{3}\frac{V}{h^3}\qty(2mE)^{3/2} \label{eq:given-states}
\end{equation}

and from GT 4.17, the number of microstates in the energy interval $\qty[E, E+\Delta E]$ is

\begin{equation}
	g(E)\Delta E = \Gamma(E+\Delta E) - \Gamma(E) \approx \dv{\Gamma(E)}{E}\Delta E \label{eq:given-interval}
\end{equation}

The mean energy of a gas molecule is given by

\begin{equation}
	E = \frac{3}{2}kT \label{eq:mean-energy}
\end{equation}

If we consider room temperature to be $T = 25^\circ C = 298 K$, then \eqref{eq:mean-energy} becomes

\begin{align}
	E &= \frac{3}{2}\qty(1.38\times 10^{-23} \mathrm{m^2\cdot kg\cdot s^{-2}\cdot K^{-1}})\qty(298 \mathrm{K}) \nonumber \\
	&\approx 6.17\times 10^{-21} \mathrm{J} \label{eq:answer-energy}
\end{align}

We consider diatomic nitrogen since it comprises around 3/4 of the atmosphere. Its molar mass is 28.02 g$\cdot$mol$^{-1}$. The mass for a nitrogen molecule is then given by

\begin{align}
	m &= \frac{\nu}{N_A} \nonumber \\
	&= \frac{28.02\mathrm{g}}{6.022\times 10^{23}} \nonumber \\
	&= 4.65\times 10^{-26} \mathrm{kg} \label{eq:answer-mass}
\end{align}

From \eqref{eq:given-interval}, we differentiate \eqref{eq:given-states} w.r.t. $E$:

\begin{align}
	\dv{\Gamma(E)}{E}\Delta E &= \frac{4\pi V}{3h^3}\qty(2m)^{3/2} \frac{3}{2}E^{1/2} \Delta E \nonumber \\
	&= \frac{2\pi V}{h^3}\qty(2m)^{3/2}\sqrt{E}\Delta E \nonumber
\end{align}

If we consider an energy interval $\Delta E = 10^{-27}$ J, this becomes

\begin{equation}
	g(E)\Delta E = \frac{2\pi\qty(1\mathrm{ L})\qty[2\qty(4.65\times 10^{-26}\mathrm{ kg})]^{3/2}\sqrt{6.17\times 10^{-21}\mathrm{ J}}}{\qty(6.63\times 10^{-34}\mathrm{ m^2\cdot kg\cdot s^{-1}})^3}\qty(10^{-27}\mathrm{ J})
\end{equation}

\begin{equation}
	\boxed{
		g(E)\Delta E \approx 4.80\times 10^{22}
	} \label{eq:answer}
\end{equation}

\end{document}