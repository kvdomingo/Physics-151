\documentclass[12pt,a4paper,twocolumn]{article}
% The following LaTeX packages must be installed on your machine: amsmath, authblk, bm, booktabs, caption, dcolumn, fancyhdr, geometry, graphicx, hyperref, latexsym, natbib
\input{151.dat}
\usepackage{gensymb}
\usepackage{amsthm}
\usepackage{float}
\usepackage{siunitx}
\usepackage{amssymb}
\usepackage{float}
\usepackage{enumerate}
\usepackage{listings}
\usepackage{mathtools}
\PassOptionsToPackage{hyphens}{url}\usepackage{hyperref}
\usepackage[none]{hyphenat}
\usepackage{physics}
%\renewcommand{\familydefault}{\sfdefault}


\begin{document}

\setcounter{page}{1}

\section*{PS 36: Problem 4.5}
\bigskip

\begin{table}[h!]
	\centering
	\begin{tabular}{|c|c||c|c|c|}
		\hline
		$E_A$ & $\Omega_A(E_A)$ & $\Omega_B(6-E_A)$ & $\Omega_A\Omega_B$ & $P_A(E_A)$ \\ \hline
		6 & 7 & 1 & 7 & 7/84 \\ \hline
		5 & 6 & 2 & 12 & 12/84 \\ \hline
		4 & 5 & 3 & 15 & 15/84 \\ \hline
		3 & 4 & 4 & 16 & 16/84 \\ \hline
		2 & 3 & 5 & 15 & 15/84 \\ \hline
		1 & 2 & 6 & 12 & 12/84 \\ \hline
		0 & 1 & 7 & 7 & 7/84 \\ \hline
	\end{tabular}
	\caption{The probability $P_A(E_A)$ that subsystem $A$ has energy $E_A$ given $N_A = N_B = 2$ and $E_{tot} = E_A + E_B = 6$.}
	\label{tab:given}
\end{table}

The standard deviation of the energy of subsystem A is given by

\begin{equation}
	\sigma = \sqrt{\ev{E_A^2} - \ev{E_A}^2} \label{eq:sd}
\end{equation}

The first moment (mean) $\ev{E_A}$ is obtained by

\begin{align}
	\ev{E_A} &= \sum_{i=1}^N (E_A)_i (P_A)_i \label{eq:mean} \\
	&= 6\qty(\frac{7}{84}) + 5\qty(\frac{12}{84}) + 4\qty(\frac{15}{84}) \nonumber \\
	&+ 3\qty(\frac{16}{84}) + 2\qty(\frac{15}{84}) + 1\qty(\frac{12}{84}) + 0\qty(\frac{7}{84}) \nonumber \\
	&= 12 \nonumber
\end{align}

The second moment (raw variance) $\ev{E_A^2}$ is obtained by

\begin{align}
	\ev{E_A^2} &= \sum_{i=1}^N (E_A)_i^2 (P_A)_i \label{eq:var} \\
	&= 36\qty(\frac{7}{84}) + 25\qty(\frac{12}{84}) + 16\qty(\frac{15}{84}) \nonumber \\
	&+ 9\qty(\frac{16}{84}) + 4\qty(\frac{15}{84}) + 1\qty(\frac{12}{84}) + 0\qty(\frac{7}{84}) \nonumber \\
	&= 3 \nonumber
\end{align}

So the standard deviation is

\begin{align}
	\sigma &= \sqrt{12 - 3^2} \nonumber \\
	&= \sqrt{3} \nonumber \\
	\Aboxed{
		\sigma &\approx 1.73
	}
\end{align}

\end{document}