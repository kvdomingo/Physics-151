\documentclass[12pt,a4paper]{article}
% The following LaTeX packages must be installed on your machine: amsmath, authblk, bm, booktabs, caption, dcolumn, fancyhdr, geometry, graphicx, hyperref, latexsym, natbib
\input{151.dat}
\usepackage{gensymb}
\usepackage{amsthm}
\usepackage{float}
\usepackage{siunitx}
\usepackage{amssymb}
\usepackage{float}
\usepackage{enumerate}
\usepackage{listings}
\usepackage{mathtools}
\PassOptionsToPackage{hyphens}{url}\usepackage{hyperref}
\usepackage[none]{hyphenat}
\usepackage{physics}
%\renewcommand{\familydefault}{\sfdefault}


\begin{document}

\setcounter{page}{1}

\section*{PS 45: Problem 4.24}
\bigskip

The relative probability without normalization/partition function is given by

\begin{equation}
	P_{rel} = e^{-\beta E_s} \label{eq:given-p}
\end{equation}

The energy difference between the two conformations is $\Delta E/k = 4180$ K, with the trans isomer lower than the cis isomer. The relative abundance is given by

\begin{equation}
	P_{rel} = e^{-\Delta E/kT} \label{eq:given-prel}
\end{equation}

At $T = 300$ K, the relative abundance between the two isomers is

\begin{equation}
	\boxed{
		\eval{P_{rel}}_{T=300} = e^{-4180/300} \approx  8.89 \times 10^{-7}
	}
\end{equation}

At $T = 1000$ K,

\begin{equation}
	\boxed{
		\eval{P_{rel}}_{T=1000} = e^{-4180/1000} \approx 1.53 \times 10^{-2}
	}
\end{equation}

This indicates that at low temperatures, the relative abundance between the cis and trans conformations is negligible. At higher temperatures, the cis isomer is more abundant.

\end{document}