\documentclass[12pt,a4paper]{article}
% The following LaTeX packages must be installed on your machine: amsmath, authblk, bm, booktabs, caption, dcolumn, fancyhdr, geometry, graphicx, hyperref, latexsym, natbib
\input{151.dat}
\usepackage{gensymb}
\usepackage{amsthm}
\usepackage{float}
\usepackage{siunitx}
\usepackage{amssymb}
\usepackage{float}
\usepackage{enumerate}
\usepackage{listings}
\usepackage{mathtools}
\PassOptionsToPackage{hyphens}{url}\usepackage{hyperref}
\usepackage[none]{hyphenat}
\usepackage{physics}
\newcommand\ddfrac[2]{\frac{\displaystyle #1}{\displaystyle #2}}
%\renewcommand{\familydefault}{\sfdefault}


\begin{document}

\setcounter{page}{1}

\section*{PS 46: Problem 4.25}
\bigskip

The Boltzmann distribution is given by

\begin{equation}
	P_S = \frac{1}{Z} e^{-\beta E_s} \label{eq:boltzmann}
\end{equation}

where $Z \equiv \sum_s e^{-\beta E_s}$ is the partition function. The probability that a system is in any microstate with energy $E$ is

\begin{equation}
	p(E) = \ddfrac{\Omega(E) e^{-\beta E}}{\sum_{\mathrm{levels}} \Omega(E) e^{-\beta E}} \label{eq:energy}
\end{equation}

In the limit $N$, $V \rightarrow \infty$, the gap between adjacent energy levels becomes infinitesimal, so $E$ can be considered a continuous variable. The probability that a system is in any microstate with energy between $E$ and $E + \dd{E}$ is $p(E)\dd{E}$. Let $g(E)\dd{E}$ be the number of microstates between $E$ and $E + \dd{E}$. The form of the probability distribution of the energy of a system in the canonical ensemble is

\begin{equation}
	\boxed{
		p(E)\dd{E} = \ddfrac{g(E) e^{-\beta E} \dd{E}}{\int_0^\infty g(E) e^{-\beta E} \dd{E}}
	} \label{eq:answer}
\end{equation}

where $\beta \equiv \ddfrac{1}{kT}$.

\end{document}