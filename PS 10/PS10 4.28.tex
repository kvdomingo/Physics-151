\documentclass[11pt,a4paper,twocolumn]{article}
% The following LaTeX packages must be installed on your machine: amsmath, authblk, bm, booktabs, caption, dcolumn, fancyhdr, geometry, graphicx, hyperref, latexsym, natbib
\input{151.dat}
\usepackage{gensymb}
\usepackage{amsthm}
\usepackage{float}
\usepackage{siunitx}
\usepackage{amssymb}
\usepackage{float}
\usepackage{enumerate}
\usepackage{listings}
\usepackage{mathtools}
\PassOptionsToPackage{hyphens}{url}\usepackage{hyperref}
\usepackage[none]{hyphenat}
\usepackage{physics}
\newcommand\ddfrac[2]{\frac{\displaystyle #1}{\displaystyle #2}}
%\renewcommand{\familydefault}{\sfdefault}


\begin{document}

\setcounter{page}{1}

\section*{PS 48: Problem 4.28}
\bigskip

\begin{enumerate}[(a)]

\item The energy levels of a single 1D harmonic oscillator is given by

\begin{equation}
	\epsilon_n = \qty(n + \frac{1}{2})\hbar\omega \label{eq:energy-levels}
\end{equation}

and the corresponding partition function is

\begin{align*}
	Z_1 &= \sum_{n=0}^\infty e^{-\beta\hbar\omega\qty(n+\frac{1}{2})} \\
	&= e^{-\frac{1}{2}\beta\hbar\omega} \sum_{n=0}^\infty e^{-n\beta\hbar\omega} \\
	&= e^{-\frac{1}{2}\beta\hbar\omega} \sum_{n=0}^\infty \qty(e^{-\beta\hbar\omega})^n
\end{align*}

For $x < 1$, we can use the identity

\begin{equation} \label{eq:geometric-identity}
	\sum_{n=0}^\infty x^n = \frac{1}{1-x}
\end{equation}

so that

\begin{equation}
	Z_1 = \frac{e^{-\frac{1}{2}\beta\hbar\omega}}{1 - e^{-\beta\hbar\omega}} \label{eq:partition}
\end{equation}

The free energy per particle is given by

\begin{align}
	f &= -kT \ln{Z_1} \label{eq:free-energy} \\
	&= -kT \ln\qty(\frac{e^{-\frac{1}{2}\beta\hbar\omega}}{1 - e^{-\beta\hbar\omega}}) \nonumber \\
	&= -kT \qty[\ln\qty(e^{-\frac{1}{2}\beta\hbar\omega}) - \ln\qty(1 - e^{-\beta\hbar\omega})] \nonumber \\
	&= \frac{1}{2}kT\beta\hbar\omega + kT\ln\qty(1 - e^{-\beta\hbar\omega})
\end{align}

We use the fact that $\beta = \ddfrac{1}{kT}$ so that

\begin{equation}
	\boxed{
		f = \frac{1}{2}\hbar\omega + kT\ln\qty(1 - e^{-\beta\hbar\omega})
	} \qed \label{eq:answer-f}
\end{equation}

The mean energy per particle is given by

\begin{align}
	\bar{e} &= -\pdv{\beta}\ln{Z_1} \label{eq:mean-energy} \\
	&= -\pdv{\beta}\ln\qty(\frac{e^{-\frac{1}{2}\beta\hbar\omega}}{1 - e^{-\beta\hbar\omega}}) \nonumber \\
	&= -\pdv{\beta}\qty[\ln\qty(e^{-\frac{1}{2}\beta\hbar\omega}) - \ln\qty(1 - e^{-\beta\hbar\omega})] \nonumber \\
	&= \frac{1}{2}\hbar\omega + \frac{e^{-\beta\hbar\omega}}{1 - e^{-\beta\hbar\omega}}\hbar\omega \nonumber \\
	\Aboxed{
		\bar{e} &= \hbar\omega\qty[\frac{1}{2} + \frac{1}{1 - e^{-\beta\hbar\omega}}]
	} \qed \label{eq:answer-e}
\end{align}

The entropy per particle is given by

\begin{align}
	s &= -\qty(\pdv{f}{T})_V \label{eq:entropy} \\
	&= -\pdv{T}\qty[\frac{1}{2}\hbar\omega + kT\ln\qty(1 - e^{-\beta\hbar\omega})]_V \nonumber \\
	&= -\pdv{T}\qty[\frac{1}{2}\hbar\omega + kT\ln\qty(1 - e^{-\hbar\omega/kT})]_V \nonumber \\
	&= \frac{1}{T}\frac{\hbar\omega e^{-\hbar\omega/kT}}{1-e^{-\hbar\omega/kT}} - k\ln\qty(1 - e^{-\hbar\omega/kT}) \nonumber \\
	&= k\frac{\beta\hbar\omega e^{-\beta\hbar\omega}}{1-e^{-\beta\hbar\omega}} - k\ln\qty(1 - e^{-\beta\hbar\omega}) \nonumber \\
	&= k\frac{\beta\hbar\omega e^{-\beta\hbar\omega}}{1-e^{-\beta\hbar\omega}} \cdot \frac{e^{\beta\hbar\omega}}{e^{\beta\hbar\omega}} - k\ln\qty(1 - e^{-\beta\hbar\omega}) \nonumber \\
	&= k\frac{\beta\hbar\omega}{e^{\beta\hbar\omega} - 1} - k\ln\qty(1 - e^{-\beta\hbar\omega}) \nonumber \\
	\Aboxed{
		s &= k\qty[\frac{\beta\hbar\omega}{e^{\beta\hbar\omega} - 1} - \ln\qty(1 - e^{-\beta\hbar\omega})]
	} \qed \label{eq:answer-s}
\end{align}

\item From \eqref{eq:answer-e}, the mean energy of a system of $N$ harmonic oscillator in equilibrium with a heat bath at temperature $T$ is

\begin{equation}
	\boxed{
		\bar{e} = N\hbar\omega\qty[\frac{1}{2} + \frac{1}{1 - e^{-\hbar\omega/kT}}]
	} \label{eq:answer-b}
\end{equation}

\item The answer differs from the result for the energy of $N$ harmonic oscillators calculated in the microcanonical ensemble in Problem 4.22:

\begin{equation}
	E(T) = \frac{N\hbar\omega}{2}\qty(\frac{e^{\hbar\omega/kT} + 1}{e^{\hbar\omega/kT} - 1}) \label{eq:answer-c}
\end{equation}

\end{enumerate}

\end{document}